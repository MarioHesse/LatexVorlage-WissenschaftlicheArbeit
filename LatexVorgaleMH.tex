% !TeX spellcheck = de_DE

%  ******************************************************************************
%  * @file      LatexVorlageMH                                                  *
%  * @author    Mario Hesse                                                     *
%  * @version   v2.0.1                                                          *
%  * @date      31.01.2019                                                      *
%  * @brief     Vorlage für wissenschaftliche Arbeiten.                         *
%  *            Erstellt von Mario Hesse                                        *
%  *                                                                            *
%  *            Alle Kapitel, die Präambel, Bilder, Literatur, etc. sind in     *
%  *            einzelnen Ordnern hinterlegt. Dadurch entsteht mehr Übersicht   *
%  *            und Ordnung.                                                    *
%  *              - Präambel:   /tex/                                           *
%  *              - Glossar:    /tex/                                           *
%  *              - Kapitel:    /tex/                                           *
%  *              - Bilder:     /pic/                                           *
%  *              - Daten:      /dat/                                           *
%  *              - Anhang:     /app/                                           *
%  ******************************************************************************

% Alle Dateien mit den nachfolgenden Endungen sind Hilfsdateien und können regelmäßig gelöscht werden. Dadurch können Fehlerhafte Dokumente oft wieder Lauffähig werden. 
% log, aux, dvi, lof, lot, bit, idx, glo, bbl, bcf, ilg, toc, ind, out, blg, fdb_latexmk, fls, acn, acr, alg, glg, gls, ist, slg, syg, syi, synctex.gz, tdo


% !TeX spellcheck = de_DE

%  ******************************************************************************
%  * @file		tex/preamble                 							        *
%  * @author	Mario Hesse                       								*
%  * @version	v0.1.0                            								*
%  * @date		31.01.2019                        								*
%  ******************************************************************************


% Dokument / Seiten
% -----------------
\documentclass[12pt,a4paper,parskip=full+]{scrartcl}
\usepackage[utf8]{inputenc}				% Zeichencodierung
\usepackage[ngerman,english]{babel}				% Sprachpaket Deutsch ausgewählt
\selectlanguage{ngerman}		% Sprache auf Deutsch gesetzt
\usepackage[T1]{fontenc}				% Umlaute bei \hyphenation verwenden können (Silbentrennung)
\usepackage[onehalfspacing]{setspace}	% Setzt das Dokument auf 1,5x Zeilenabstand (singlespacing/onehalfspacing/doublespacing)
\usepackage[headsepline]{scrlayer-scrpage}		% Kopfzeile
\usepackage[defaultlines=4,all]{nowidow}				% Schusterjungen und Hurenkinder
\usepackage{							% standard Krempel
	amsmath,							% American Mathematical Society - Mathematik Basics (equation, equation*, align, align*, 
										% gather, gather*, flalign, flalign*, multline, multline*, alignat, alignat*, ...)
	amsfonts,							% Schriftarten für Mathematik und spezielle Symbole
	amsthm,								% erlaubt die Definiton von Theoremen
	amssymb}							% Mathematische Zeichen und Symbole



% Korrektur- und Dokumentatitonswerkzeuge
% ---------------------------------------
\usepackage{comment}					% Erlaubt das Verwenden der Umgebung \begin{comment}...\end{comment} -> Erstellen von Kommentaren über mehrere Zeilen
\usepackage[ngerman,
			disable,
%			textwidth=5cm,
			textsize=tiny,
			colorinlistoftodos,
			]{todonotes}				% Erlaubt das Verwenden von ToDo-Notitzen -> \todo{...} und Blindbildern -> \missingfigure{...}

\usepackage[ngerman]{translator}		% Erlaubt anderen Paketen Begriffe in Deutsche zu Übersetzen (Figure -> Abbildung, Table -> Tabelle, etc.)

\setlength{\headheight}{1.1\baselineskip}
\newcommand{\mario}[2][]{\todo[color=green,#1]{\textbf{mh:} #2}}
%\newcommand{\korrektor1}[1]{\todo[color=blue!40]{\textbf{ug:} #1}}
\newcommand{\wolfgang}[2][]{\todo[color=purple!30,#1]{\textbf{wv:} #2}}


% Gleitumgebungen / Grafiken
% --------------------------
\usepackage{float}						% notwendig für Gleitubgebungen z.B. Figure
\usepackage{subfig}						% erlaubt das Verwenden von Subfigures
\usepackage{graphicx}					% erlaubt das Einbinden von Grafiken
\usepackage{pdfpages}					% ermöglicht das Einbinden von PDFs
\usepackage{epstopdf}					% einbinden von eps Bilddateien
\usepackage{afterpage}					% Im­ple­ments a com­mand that causes the com­mands spec­i­fied in its ar­gu­ment to be ex­panded af­ter the cur­rent page is out­put
\usepackage{tabularx}					% ermöglicht Tabbellen mit konkreter Breite festzulegen und erlaubt Zeilenumbruch in einer Tabelle

\newcolumntype{L}[1]{>{\raggedright\arraybackslash}p{#1}} 	% linksbündig mit Breitenangabe -> L{BREITE} statt p{BREITE} möglich
\newcolumntype{C}[1]{>{\centering\arraybackslash}p{#1}} 	% zentriert mit Breitenangabe -> C{BREITE} statt p{BREITE} möglich
\newcolumntype{R}[1]{>{\raggedleft\arraybackslash}p{#1}} 	% rechtsbündig mit Breitenangabe -> R{BREITE} statt p{BREITE} möglich
\newcolumntype{Y}{>{\centering\arraybackslash}X}			% X Spalten zentrieren
\newcolumntype{Z}{>{\raggedleft\arraybackslash}X}			% X Spalten zentrieren



\newcommand{\ltab}{\raggedright\arraybackslash} 	% Tabellenabschnitt linksbündig -> \ltab
\newcommand{\ctab}{\centering\arraybackslash} 		% Tabellenabschnitt zentriert -> \ctab 
\newcommand{\rtab}{\raggedleft\arraybackslash} 		% Tabellenabschnitt rechtsbündig -> \rtab


\usepackage{longtable}					% lange Tabellen verarbeiten
\usepackage{caption}					% ermöglicht die Beschriftung von Gleitobjekten
\usepackage{nameref}					% erlaubt Namensreferenzen
\usepackage{wrapfig}                    % Text umschließt Figure

\usepackage{forest}						% erlaubt einfache Verzeichnisstrukturen (tree)

% Diagramme
% ---------
\usepackage{tikz}						% Zeichentool
\usepackage{pgfplots, pgfplotstable}	% Diagramme zeichnen
\pgfplotsset{compat = newest}			% Einstellung für Diagramme
\usepackage[binary-units]{siunitx}      % SI-Einheiten
\usepgfplotslibrary{units}				% Einheiten in Diagrammen plotten


% Verzeichnisse
% -------------
\usepackage[
%	super,
	square,
	numbers
	]{natbib}	% Literatur zitieren
\bibliographystyle{IEEEtranN}				% Literatur- und Zitatstyle | Alternativen: dinat, abbrvdin, alphadin (DIN1505), plaindin (DIN1505), natdin, plain, abbrv, alpha, IEEEtran, IEEEtranN, IEEEtranSN, IEEEtranSA | Eigenen Style auf Konsole erzeugen: latex makebst |


% Inhaltsverzeichnis
\usepackage{makeidx}					% anlegen eines automatischen Inhaltsverzeichnis
\usepackage{tocstyle}					% Inhaltsverzeichnis
\usetocstyle{allwithdot} 				% Punkte im Inhaltsverzeichnis
\setcounter{tocdepth}{2}				% Inhaltsverzeichnis Ebenen definieren, zB bis zu Subsection


% Verlinkungen
% ------------
\usepackage[							% Verlinkung aller sections, ref , url, etc.
	pdfborderstyle={/S/U/W 1}				% Ändert den Rahmen der Links in Unterstriche
	% colorlinks=true,						% Schaltet Rahmen um die Links aus und ändert statt dessen die Schriftfarbe
	% urlcolor=blue,						% URL Links: Blau
	% citecolor=black,						% Zitate: Schwarz
	% linkcolor=black
	]{hyperref}		
\hypersetup{								% Erweiterte Einstellungen zu "hyperref" und erlaubt \autoref ()ist ähnlich wie \ref, schreibt aber Abbildung, Tabelle, usw. vor die verlinkte Gleitumgebung)
	bookmarks=true,							% Zeigt die Lesezeichenleiste an oder nicht.
	unicode=false,							% Ermöglicht die Verwendung von nicht-latin Schriftzeichen
	pdftoolbar=true,						% Setzt den Rahmen um einen Link. {0 0 0}erzeugt keinen Rahmen
	pdfmenubar=true,						% Zeigt die Acrobat-Toolbar an oder versteckt sie.
	pdffitwindow=false,						% Zeigt das Acrobat-Menü an oder versteckt es.
	pdfstartview={FitH},					% Verändert die Größe des Acrobat-Anzeigefensters, damit das Dokument hineinpasst.
	pdftitle={Exposé zur Promotion: Intelligentes Plasma},						% Definiert den Titel des Dokumentes, welcher in der Dokumenteninfo von Acrobat angezeigt wird.
	pdfauthor={Mario Hesse},						% Definiert den Namen des Autors für die Dokumenteninfo.
	pdfsubject={},					% Beschreibung des Dokument für die Dokumenteninfo.
	pdfcreator={Mario Hesse},				% Ersteller des Dokuments für die Dokumenteninfo.
	pdfproducer={Mario Hesse},							% Produzent des Dokuments für die Dokumenteninfo.
	pdfkeywords={Intelligentes Plasma} {Kaltes Plasma} {Atmosphärendruck Plasma} {DBD Plasma} {Plasma Regelung} {Plasma Steuerung} {Plasma Automation} {Hochschule für angewandte Wissenschaft und Kunst} {HAWK} {Plasma for Life} {Göttingen},		% Schlüsselwörter des Dokuments für die Dokumenteninfo (durch geschweifte Klammern voneinander getrennt; siehe unten).
	pdfnewwindow=true,						% Legt fest, dass Links in einem neuen Fenster geöffnet werden sollen.
	colorlinks=false,						% Legt fest, ob ein farbiger Rahmen um die Links gezogen werden soll oder ob die Schrift farbig sein soll.
	linkcolor=red,							% Farbe für die Links.
	citecolor=green,						% Farbe für die Quelllinks (bibliography; Quellenverzeichnis).
	filecolor=magenta,						% Farbe für Dateilinks.
	urlcolor=cyan							% Farbe für URL-Links (Web, Mail).
}

\renewcaptionname{ngerman}\sectionautorefname{Kapitel}			% Ändern von \autoref(section): Abschnitt 			-> Abschnitt
\renewcaptionname{ngerman}\subsectionautorefname{Kapitel}			% Ändern von \autoref(section): Unterabschnitt 		-> Abschnitt
\renewcaptionname{ngerman}\subsubsectionautorefname{Kapitel}		% Ändern von \autoref(section): Unterunterabschnitt	-> Abschnitt
\newcommand{\subfigureautorefname}{\figureautorefname}		% Ändern von \autoref(subfloat): ~	-> tt

\usepackage{cleveref}						% Muss nach \hypreref geladen werden (!) \cref ist ähnlich wie \ref, schreibt aber Abb., Tab., usw. vor die verlinkte Gleitumgebung


% Glossar, Abkürzungsverzeichnis, Symbolverzeichnis, etc.
% -------------------------------------------------------
\usepackage[nonumberlist, 											% keine Seitenzahlen anzeigen
			acronym,      											% ein Abkürzungsverzeichnis erstellen
			toc,          											% Einträge im Inhaltsverzeichnis
			section      											% im Inhaltsverzeichnis auf section-Ebene erscheinen
			]{glossaries}										% Erlaubt Glossar, Abkürzungsverzeichnis, Symbolverzeichnis und mehr...
\newglossary[slg]{symbolslist}{syi}{syg}{Symbolverzeichnis}		% Ein eigenes Symbolverzeichnis erstellen
\renewcommand*{\glspostdescription}{}							% Den Punkt am Ende jeder Beschreibung deaktivieren
\newcommand{\glsit}[1]{\textit{\gls{#1}}}						% Neues Kommando \glsit{} = \gls{} mit Link als kursivem Text
\newcommand{\glspit}[1]{\textit{\glspl{#1}}}					% Neues Kommando \glsplit{} = \glpl{} mit Link als kursivem Text
\makeglossaries													% Glossar-Befehle anschalten


% Formatierungen
% --------------
\usepackage{abstract}					% erlaubt des Verwenden der standartisierten Vorlage für Abstracts ( \begin{abstract}...\end{abstract} )
\addto\captionsngerman{%				% benennt den deutschen Titel des Abstracts von 'Zusammenfassung' um in 'Kurzfassung' 
	\renewcommand{\abstractname}{Kurzfassung}}
\usepackage{enumitem}					% ermöglicht das Einstellen von Abständen in itemize Umgebungen -> \setitemize{noitemsep,topsep=0pt,parsep=0pt,partopsep=0pt} (global) oder begin{itemize}[noitemsep,topsep=0pt,parsep=0pt,partopsep=0pt] (lokal) und erlaubt das Verwenden der {enumitem} Umgebung (itemize mit geringeren Abständen)
\setitemize{noitemsep,topsep=0pt,parsep=0pt,partopsep=0pt}
\usepackage{eurosym}					% €-Zeichen mit \euro darstellbar oder durch \EUR{betrag}
\usepackage{fdsymbol}					% Symboltabelle mit \hateq (Entspricht Symbol)
\usepackage{romannum}					% Römische Nummern schreiben mit '\romannum{integer}'
\newcommand{\Rom}[1]{\uppercase\expandafter{\romannumeral #1\relax}}		% Römische Nummern schreiben mit '\Rom{integer}' (nötig für Verwendung in \section{})
\usepackage{textgreek}					% Erlaubt grichische Buchstaben im Text mit \textalpha, \textbeta, ...

\setlength{\parindent}{0in} 			% Absatz, erste Zeile wird global NICHT eingerückt
%\usepackage{parskip}					% setzt einen Abstand nach einem \par Absatz
\usepackage{rotating}					% erlaubt das rotieren eines Elements
\newcommand\tabrotate[1]{\begin{turn}{90}\rlap{#1}\end{turn}}	% kurzbefehl für das Rotieren eines Elements

\usepackage{multirow}					% erlaubt in Tabellen das Zusammenfassen von Zellen in Reihe
\usepackage{booktabs}					% erlaubt mehrere unterschiedliche Linien in Tabellen (\toprule, \midrule, \bottomrule)
\usepackage{hhline}						% erlaubt individuelle Dicke bei Tabellenlinien

\usepackage{csquotes}
\newcommand{\gqt}[1]{\glqq #1\grqq{}}	% \gqt{...} (german quotation) ist eine Umgebung für deutsche Anführungszeichen | Alternativen: dirtytalk, csquotes, epigraph

% Custom colors
\usepackage{color}
\definecolor{deepblue}{rgb}{0,0,0.5}	% definiert ein tiefes Blau
\definecolor{deepred}{rgb}{0.6,0,0}		% definiert ein tiefes Rot
\definecolor{deepgreen}{rgb}{0,0.5,0}	% definiert ein tiefes Grün


\usepackage{wasysym}					% zusätzliche Symbole verwenden (Checkbox)
\usepackage{listings}					% ermölgicht Zeichengenaues Zitieren (für Programmcode) \verb|Zitat|


% Silbentrennung
%\expandafter\def\expandafter\UrlBreaks\expandafter{\UrlBreaks\do\a		% Erlaubt einen Umbruch in einer URL 
%	\do\b\do\c\do\d\do\e\do\f\do\g\do\h\do\i\do\j\do\k\do\l\do\m\do\n	% an allen angegebenen Stellen
%	\do\o\do\p\do\q\do\r\do\s\do\t\do\u\do\v\do\w\do\x\do\y\do\z\do\A	
%	\do\B\do\C\do\D\do\E\do\F\do\G\do\H\do\I\do\J\do\K\do\L\do\M\do\N	
%	\do\O\do\P\do\Q\do\R\do\S\do\T\do\U\do\V\do\W\do\X\do\Y\do\Z\do\1
%	\do\2\do\3\do\4\do\5\do\6\do\7\do\8\do\9\do\0\do\&}
\expandafter\def\expandafter\UrlBreaks\expandafter{\UrlBreaks
	\do.\do-\do:}													% nur an markanten Stellen

%\sloppy								% Nachlässige Silbentrennung, Blocksatzoptimiert
\hyphenation{OpenOCD To-po-lo-gie-in-for-ma-ti-onen Fer-ti-gungs-au-to-ma-ti-on Kom-mu-ni-ka-tions-ein-heit CANopen-Kom-mu-ni-ka-tions-ein-heit Atmos-phä-ren-druck ei-des-statt-li-ch UART USART El-ek-tro-den-an-ord-nung Mi-kro-plas-ma-zel-len Mi-kro-plas-ma-zel-len}					% Worte, die NICHT oder nur auf bestimmte Weise getrennt werden sollen

% \itemize mit Überschrift
\newenvironment{titlemize}[1]{
	\paragraph{#1}
	\begin{itemize}}
	{\end{itemize}}

% \paragraph mit Zeilenumbruch versehen 
% -begin-----------------------------------------------------------
\makeatletter
\renewcommand\paragraph{
	\@startsection{paragraph}{4}{\z@}
	{-3.25ex\@plus -1ex \@minus -.2ex}			% Raum vor \paragraph
	{1.5ex \@plus .2ex}							% Raum nach \paragraph
	{\normalfont\normalsize\bfseries}			% Text unter \paragraph
}
\makeatother
% -end-------------------------------------------------------------


% Python
% -begin----------------------------------------------------------
% Include Python-Programmcode with Syntax-Highlighting
% Default fixed font does not support bold face
\DeclareFixedFont{\ttb}{T1}{txtt}{bx}{n}{11} % for bold
\DeclareFixedFont{\ttm}{T1}{txtt}{m}{n}{11}  % for normal

% Python style for highlighting
\newcommand\pythonstyle{\lstset{
language=Python,
basicstyle=\ttm,
otherkeywords={self},             % Add keywords here
keywordstyle=\ttb\color{deepblue},
emph={MyClass,__init__},          % Custom highlighting
emphstyle=\ttb\color{deepred},    % Custom highlighting style
stringstyle=\color{deepgreen},
frame=tb,                         % Any extra options here
showstringspaces=false            % 
}}


\lstnewenvironment{python}[1][]		% Python environment
{
\pythonstyle
\lstset{#1}
}
{}

% Python for external files
\newcommand\pythonexternal[2][]{{
\pythonstyle
\lstinputlisting[#1]{#2}}}

% Python for inline
\newcommand\pythoninline[1]{{\pythonstyle\lstinline!#1!}}
% -end------------------------------------------------------------


% Citation aliases
%\defcitealias{2011IndustrialEthernetFacts}{Industrial Ethernet Facts [2011]}



% Fehler bekämpfen
\pdfoptionpdfminorversion=7


% !TeX spellcheck = de_DE
% !TeX encoding = UTF-8

%  ***********************************************
%  * Dokument: tex/Glossaries           *
%  * Editor:   Mario Hesse                       *
%  * Version:  v0.1.0                           *
%  * Date:     31.01..2019                        *
%  ***********************************************

% Symbole
% -------
\newglossaryentry{symb:U}{
	name=$U$,
	description={Spannung},
	sort=symbolU,
	type=symbolslist
}
\newglossaryentry{symb:I}{
	name=$I$,
	description={Strom},
	sort=symbolI,
	type=symbolslist
}
\newglossaryentry{symb:R}{
	name={$R$},
	description={Widerstand},
	sort=symbolR,
	type=symbolslist
}
\newglossaryentry{symb:C}{
	name={$C$},
	description={Kapazität},
	sort=symbolC,
	type=symbolslist
}
\newglossaryentry{symb:f}{
	name={$f_g$},
	description={Frequenz},
	sort=symbolf,
	type=symbolslist
}
\newglossaryentry{symb:pi}{
	name={$\pi$},
	description={Kreiszahl ($\approx$3,142)},
	sort=symbolpi,
	type=symbolslist
}


 
% Abkürzungen
% -----------
\newacronym
	{ems}{EMS}{elektromagnetische Störung}
\newacronym
	{emv}{EMV}{elektromagnetische Verträglichkeit}
\newacronym
	{sps}{SPS}{\gls{SpeicherprogrammierbareSteuerung}}
\newacronym
	{hawk}{HAWK}{Hochschule für angewandte Wissenschaft und Kunst Hildesheim/Holzminden/Göttingen}
\newacronym[description={Analog-Digital-Converter (Analog-Digital-Wandler)}]
	{adc}{ADC}{Analog-Digital-Converter}


 
% Glossar
% -------
\newglossaryentry{glos:dbd}{
	name={Dielektrisch behinderte Entladung},
	description={ist eine besondere Form der \gls{Plasma} Entladung, bei der ein kaltes Plasma mit sehr niedriger Energie auch bei Atmosphärendruck gezündet werden kann. Sie zeichnet sich durch ein Dielektrikum in der Hochspannungsmasche aus.}
}
\newglossaryentry{Steuerung}{
	name={Steuerung},
	description={beschreibt einen Prozess, der eine Größe (z.B. eine Plasmaentladung) ausgibt. Die Größe wird mit festen Werten von einem Anwender eingestellt.}
}
\newglossaryentry{Regelung}{
	name={Regelung},
	description={beschreibt einen Prozess, der eine Größe (z.B. eine Plasmaentladung) ausgibt. Der Effekt, den diese Größe hervorruft, wird gemessen und in den Prozess zurückgekoppelt, um damit die Größe so zu verändern, dass der gemessene Effekt dem Anwenderwunsch entspricht und konstant gehalten wird.}
	}
\newglossaryentry{Mikrocontroller}{
	name={Mikrocontroller},
	description={ist ein integrierter Schaltkreis, der alle elementaren Funktionseinheiten eines Computersystems auf einem Chip vereint. Heutzutage bildet er in den meisten handelsüblichen elektronischen Geräten die intelligente Steuereinheit.}
}
\newglossaryentry{SpeicherprogrammierbareSteuerung}{
	name={Speicherprogrammierbare Steuerung},
	description={ist eine Art Industriecomputer, der dazu geschaffen ist große, mittlere und kleine Industrieanlagen zu steuern.},
	plural={Speicherprogramierbare Steuerungen}
}
\newglossaryentry{Plasma}{
	name={Plasma},
	description={ist der 4. Aggregatzustand. Er entsteht, wenn einem Gas zusätzlich Energie zugeführt wird. In diesem Zustand wandeln sich Atome und Moleküle ständig in Ionen und Elektronen um und umgekehrt.}
}






												% Paket Glossaries erstellt ein Glossar, Abkürzungsverzeichnisse und Formelverzeichnisse: Für Fortgeschrittene Benutzer

% ******************************************************************************
\begin{document}
% ******************************************************************************
	\selectlanguage{ngerman}													% Sprache für Makros: Deutsch
	
	
	% Einstellen der Kopf- und Fußzeile
	% ---------------------------------
	\ohead{\pagemark}
	\ihead{\sectionmark}
	\cfoot{}
	\pagestyle{scrheadings}
	\renewcommand{\sectionmark}[1]{\markright{#1}{}}
	
	
	% ToDo Liste
	% ----------
	\listoftodos																% mit \todo{} können Randnotizen erstellt werden. Alle Randnotizen werden in der "List of Todos" an dieser Stelle angezeigt.
	
	% Titelseite
	% ----------
	% !TeX spellcheck = de_DE

%  ******************************************************************************
%  * @file      tex/Deckblatt                                                   *
%  * @author    Mario Hesse                                                     *
%  * @version   v0.1.1                                                          *
%  * @date      22.10.2019                                                      *
%  ******************************************************************************

\begin{titlepage}
	\begin{center}
		{\Large \textbf{Ein ziemlich guter wissenschaftlicher Arbeitstitel}}
		
		\vspace{\fill}
		
		Bachelorabschlussarbeit / Masterabschlussarbeit \\
		von \\
		Herrn / Frau Vorname Name
		
		\vspace{0.9cm}
		
		im Studiengang \dots \\
		mit Schwerpunkt \dots (optional)\\
		
		\vspace{0.9cm}
		
		an der HAWK Hochschule für angewandte Wissenschaft und Kunst\\
		Hildesheim / Holzminden / Göttingen\\
		Fakultät Naturwissenschaft und Technik in Göttingen
		
		\vspace{0.1cm}
		
		\includegraphics[height=3.cm]{pic/hawk-fn-logo}
		
		\vspace{0.1cm}
		
		in Kooperation mit der Firma \dots (optional)\\
		(Ohne Logo des Kooperationspartners)
		
		\vspace{2.0cm}
		
		\begin{tabular}{p{3cm}l}
			Erstprüfer: & Prof. Dr. habil. \dots \\ 
			Zweitprüfer: & M. Sc. \dots \\  
		\end{tabular} 
		
		\vspace{1.6cm}
	
		\today
		
	\end{center}
	
	
	
\end{titlepage}

	% **************************************************************************
	% Vorspiel
	% **************************************************************************
	\pagenumbering{arabic}          

	
	
	% Aufgabenstellung
	% ----------------
	\clearpage 
	\phantomsection
	\addcontentsline{toc}{section}{Aufgabenstellung}
	% !TeX spellcheck = de_DE

%  ******************************************************************************
%  * @file		tex/Aufgabenstellung                 							*
%  * @author	Mario Hesse                       								*
%  * @version	v0.1.0                            								*
%  * @date		31.01.2019                        								*
%  ******************************************************************************

\section*{Aufgabenstellung}
Hier steht die Beschreibung der Arbeitsaufgabe so wie sie vom Betreuer formuliert wurde. Am besten fügt man diese hier einfach als PDF-Dokument ein.

%\includepdf[\label{sec:Aufgabenstellung}}, scale=1.0]{app/Aufgabenstellung.pdf}

	
	
	% Sperrvermerk
	% ------------
	\clearpage 
	\phantomsection
	\addcontentsline{toc}{section}{Sperrvermerk}	 	
	% !TeX spellcheck = de_DE
% !TeX encoding = UTF-8

%  ******************************************************************************
%  * @file      tex/Sperrvermerk                                                *
%  * @author    Mario Hesse                                                     *
%  * @version   v0.1.0                                                          *
%  * @date      31.01.2019                                                      *
%  ******************************************************************************

\paragraph*{Selbständigkeitserklärung}

Hiermit erkläre ich, dass ich die vorliegende Abschlussarbeit selbstständig, ohne fremde Hilfe und nur unter Verwendung der angegebenen Literatur angefertigt habe. Alle fremden, öffentlichen Quellen sind als solche kenntlich gemacht. Mir ist bekannt, dass ich für die Quellen Dritter in dieser Arbeit die Nutzungsrechte zur Verwendung in dieser Arbeit benötige. Weiterhin versichere ich, dass diese Abschlussarbeit noch keiner anderen Prüfungskommission vorgelegen hat.

\vspace{5mm}

Göttingen, den \today \hfill Unterschrift


\paragraph*{Erklärung zu Nutzungsrechten und Verwertungsrechten\footnote{Dadurch räumen Sie der HAWK ein einfaches, zeitlich unbeschränktes, unentgeltliches Nutzungsrecht nach §§ 15 Abs. 2 Nr. 2, 16, 17, 19a, 31 Abs.2 UrhG ein.}}

Ich bin hiermit einverstanden, dass von meiner Abschlussarbeit (ggf. nach Ablauf der Sperre) 1 Vervielfältigungsstück erstellt werden kann, um es der Bibliothek der HAWK zur Verfügung zu stellen und Dritten öffentlich zugänglich zu machen. Ich erkläre, dass Rechte Dritter der Veröffentlichung nicht entgegenstehen.\\

\vspace{5mm}

Göttingen, den \today \hfill Unterschrift

\paragraph*{Sperrvermerk der Abschlussarbeit}

\indent
\hspace{15mm}\Square~NEIN\footnote{Zutreffendes bitte ankreuzen \label{ftn:Zutreffendes}}  \\ 
\indent
\hspace{15mm}\Square~JA / Dauer der Sperre: \Square~3 Jahre\footref{ftn:Zutreffendes} \Square~5 Jahre\footref{ftn:Zutreffendes}

\vspace{5mm}

Göttingen, den \today \hfill Unterschrift


	
	
	% Danksagung
	% ----------
	\clearpage 
	\phantomsection
	\addcontentsline{toc}{section}{Danksagung}
	% !TeX spellcheck = de_DE

%  ******************************************************************************
%  * @file		tex/Danksagung                 							        *
%  * @author	Mario Hesse                       								*
%  * @version	v0.1.0                            								*
%  * @date		31.01.2019                        								*
%  ******************************************************************************

\section*{Danksagung}

Es gehört zum guten Ton, sich bei der Fertigstellung einer wissenschaftlichen Arbeit bei wichtigen Menschen zu bedanken. Diese Menschen können beispielsweise deine Eltern, dein Partner und dein Betreuer sein. Es gibt allerdings auch Wissenschaftler, die sich bei ihrer Katze oder ihrem Goldfisch bedanken oder simpel \textit{Danke an alle.} schreiben. In diesem Teil der Arbeit kannst du zwar wirklich alles schreiben, was du willst, allerdings ist das auch der Teil an dem andere Leute, sehen können wie du tickst. Daher rate ich dazu dieses Kapitel nicht all zu extravagant zu schreiben.
\vspace{3em}

Göttingen, den \today

	
	
%	% Vorwort																	% Optionales Kapitel
%	% -------
%	\clearpage 
%	\phantomsection
%	\addcontentsline{toc}{section}{Vorwort}
%	% !TeX spellcheck = de_DE

%  *****************************************************
%  *  Name:     tex/Vorwort                            *
%  *  Editor:   Mario Hesse                            *
%  *  Version:  v0.1.1                                 *
%  *  Date:     22.09.2017                             *
%  *****************************************************

\section*{Vorwort}

\mario[]{Kapitel \glqq Vorwort\grqq{} schreiben}

	
	
	% Kurzfassung und Abstract
	% ------------------------
	\clearpage
	\phantomsection
	\addcontentsline{toc}{section}{Kurzfassung und Abstract}
	% !TeX spellcheck = de_DE

%  ******************************************************************************
%  * @file      tex/Kurzfassung                                                 *
%  * @author    Mario Hesse                                                     *
%  * @version   v0.1.0                                                          *
%  * @date      31.01.2019                                                      *
%  ******************************************************************************


\begin{abstract}
	Kurzbeschreibung in Deutsch.
	
	\medskip
	\textit{Schlagwörter: Plasma, Atmosphärendruck, dielektrisch behinderte Entladung, Oberflächenbehandlung, ...}
\end{abstract}

\vfill			% Füllt die Seite an dieser Stelle mit Leerzeilen auf statt am Ende

\selectlanguage{english}
\begin{abstract}
	Short description in English.	
	
	\medskip
	\textit{Keywords: plasma, atmospheric pressure, dielectric barrier discharge, surface treatment, ...}
\end{abstract}
\selectlanguage{ngerman}
	

	% **************************************************************************
	% Verzeichnisse
	% **************************************************************************
	
	% Inhaltsverzeichnis
	% ------------------
	\clearpage
	\phantomsection
	\addcontentsline{toc}{section}{Inhaltsverzeichnis}
	\tableofcontents
	
	
	% Abbildungsverzeichnis
	% ---------------------
	\clearpage
	\phantomsection
	\addcontentsline{toc}{section}{Abbildungsverzeichnis}
	\listoffigures
	
	
	% Tabellenverzeichnis
	% -------------------
	\clearpage
	\phantomsection
	\addcontentsline{toc}{section}{Tabellenverzeichnis}
	\listoftables
	
	
	% Glossarie Verzeichnisse													% Paket Glossaries erstellt ein Glossar, Abkürzungsverzeichnisse und Formelverzeichnisse: Für Fortgeschrittene Benutzer
	% -----------------------
	% Glossar ausgeben
	\clearpage
	\printglossary[style=altlist,title=Glossar]
	
	% Abkürzungen ausgeben
	\clearpage
	\printglossary[type=\acronymtype,title=Abkürzungsverzeichnis,style=long]
	
	% Symbole ausgeben
	\clearpage
	\printglossary[type=symbolslist,style=long]
	
	
	
	% **************************************************************************
	% Dokumententext 
	% **************************************************************************
	\setcounter{table}{0}														% Tabellenzähler wird auf 0 gesetzt (Falls vorher schon eine Tabelle genutzt wurde)
	
	
	% Einleitung
	% ----------
	\clearpage
	\pagenumbering{arabic}                                                             
	% !TeX spellcheck = de_DE

%  ******************************************************************************
%  * @file      tex/Einleitung                                                  *
%  * @author    Mario Hesse                                                     *
%  * @version   v0.1.0                                                          *
%  * @date      31.01.2019                                                      *
%  ******************************************************************************


\section{Einleitung}

Ein paar einleitende Worte. Wo schreibe ich die Arbeit? In welchem Kontext? Wer ist daran beteiligt?


	\subsection{Motivation}
	
	Aus welchem Grund wird dieses Thema bearbeitet? Wo ist der Vorteil? Welche wissenschaftliche Frage steht dahinter? Welches bisher nicht gelöste Problem wird gelöst?
	

	\subsection{Beschreibung der Ausgangssituation}
	
	Wie wird aktuell in dem Bereich gearbeitet? An welcher Stelle setzt die Arbeit an? Welche Vorarbeit wurde bisher geleistet?
	
	
	\subsection{Zielstellung} \label{sec:Zielstellung}
	
	Welcher Zustand soll erreicht werden? An welchem Punkt kann die Aufgabenstellung als gelöst angesehen werden?
	
	
	
	
	
	
	
	\subsection{Test für Glossaries}
	
	In diesem Abschnitt kann das Paket Glossaries getestet werden. Dazu müssen im Hauptdokument zwei Stellen einkommentiert werden. Diese Stellen befinden sich in Zeile 27 und 119 - 131.
	
	\paragraph{Abkürzungen}
	
	Wenn man die \gls{hawk} in einem Text erwähnt, muss man Sie beim ersten Mal ausschreiben. Bei der zweiten Erwähnung darf man die \gls{hawk} dann abkürzen. Glossaries macht das automatisch. Um das Glossar selbst zu erstellen, darf man nicht vergessen den Befehl \textit{Makeglossaries} auszuführen. In TeXstudio findet man ihn unter \textit{Menüleiste/Tools/Befehle/Makeglossaries}. Dieser Befehl muss immer ausgeführt werden, wenn neue Glossar-Wörter in den Text kommen. 
	
	\paragraph{Glossar}
	
	Sollte ein Begriff wie \gls{Plasma} erklärungsbedürftig sein, kann man ihn in \textit{/tex/Glossaries.tex} definieren. Er landet im Glossar, sobald er mit \textit{$\backslash$gls\{KeyDesBegriffs\}} im Text erwähnt wird und die Befehle \textit{Makeglossaries} und \textit{Erstellen \& Anzeigen} ausgeführt werden.
	
	Man kann auch Verlinkungen zwischen Abkürzungen und Glossar erstellen ein schönes Beispiel hierfür sieht man im Abkürzungsverzeichnis der \gls{sps}.
	
	
	\paragraph{Formelverzeichnis}
	
	Das Formelverzeichnis wird auf gleiche Weise angelegt, wie das Glossar und das Abkürzungsverzeichnis. Sobald die Werte eingeführt werden, kann man sie im Formelverzeichnis sehen. Um das zu demonstrieren seien hier \gls{symb:U}, \gls{symb:I} und \gls{symb:R} erwähnt. In \autoref{OhmschesGesetz} kann man sehen, dass die verlinkten Formelzeichen auch in eine Gleichung eingesetzt werden können.
	
	\begin{align}
		\text{\gls{symb:R}}=\frac{\text{\gls{symb:U}}}{\text{\gls{symb:I}}} \label{OhmschesGesetz}
		\end{align}
	
	
	


	
	% Grundlagen
	% ----------
	\clearpage
	% !TeX spellcheck = de_DE

%  ******************************************************************************
%  * @file      tex/Grundlagen                                                  *
%  * @author    Mario Hesse                                                     *
%  * @version   v0.1.0                                                          *
%  * @date      31.01.2019                                                      *
%  ******************************************************************************

\clearpage

\section{Grundlagen}

Hier sollten alle Themen erklärt werden, die zum Verständnis für die Arbeit nötig sind. In diesem Kapitel werden beim Leser die theoretischen Grundlagen gelegt, mit denen er den Rest der Arbeit verstehen kann. Fachbücher \citep{Helmke2016,Metelmann2016} sind in diesem Kapitel die bevorzugte Literatur. 	
	
	% Stand der Technik
	% -----------------
	\clearpage
	% !TeX spellcheck = de_DE

%  ******************************************************************************
%  * @file		tex/StandDerTechnik                 							*
%  * @author	Mario Hesse                       								*
%  * @version	v0.1.0                            								*
%  * @date		31.01.2019                        								*
%  ******************************************************************************


\section{Stand der Technik}

In diesem Kapitel sollte der aktuelle Stand von Forschung und Technik dargestellt werden. Dazu ist es sinnvoll zu recherchieren, welche Veröffentlichungen es in dem Forschungsbereich in den letzten 2-3 Jahren gegeben hat. Die bevorzugte Literatur in diesem Teil sind Paper \citep{Peters2017,Peters2018,Bruggeman2013}, Doktor-, Bachelor- und Masterarbeiten \citep{Hirschberg2017,Freier2013}. Schau am einfachsten bei \textit{Google Scholar} oder falls ein Zugang besteht im \textit{Web of Science}. Paper aus der eigenen Arbeitsgruppe sind immer besonders schön. Auch Patente \cite{Liepack2011} sind in diesem Kapitel gern gesehen. 

Webseiten \citep{Northstar2018} sollte man als Literatur grundsätzlich vermeiden und nur dann verwenden, wenn es nicht anders geht.









	
	
	% Hardware
	% --------
	\clearpage
	% !TeX spellcheck = de_DE

%  ******************************************************************************
%  * @file      tex/Hardware                                                    *
%  * @author    Mario Hesse                                                     *
%  * @version   v0.1.0                                                          *
%  * @date      31.01.2019                                                      *
%  ******************************************************************************

\section{Hardware} \label{sec:Hardware}

In diesem Abschnitt wird Hardware beschrieben, die im Zuge der Arbeit entsteht. Dazu könnten Tabellen wie \autoref{tab:DieErsteTabelle} nützlich sein.
		
		
\begin{table}[htbp]
	\centering																	% zentriert die Tabelle
	\caption{Die erste Tabelle}													% Tabellenbezeichnung (Taucht im Tabellenverzeichnis auf)
	\label{tab:DieErsteTabelle}													% Label mit dem auf die Tabelle verwiesen werden kann
	\begin{tabular}{l|c|r}														% Die Tabellen Spalten werden mit l,c oder r als linkbündig, zentriert oder rechtsbündig definiert; soviele Buchstaben wie geschrieben wurden, soviele Spalten existieren
		links angeordnet	& zentriert angeordnet	& rechts angeordnet \\ 		% Tabelleneinträge werden durch ein "&" getrennt und jede Reihe wird duch "\\" abgeschlossen
		\hline 																	% Erzeugt eine horizontale Linie
		links 				& zentriert 			& rechts \\					% Weitere Reihen schreibt man im gleichen Stil darunter
		l 					& z			 			& r \\					% Weitere Reihen schreibt man im gleichen Stil darunter
		
	\end{tabular}
\end{table}
	
	% Firmware/Software
	% -----------------
	\clearpage
	% !TeX spellcheck = de_DE

%  ******************************************************************************
%  * @file		tex/Firmware                 							        *
%  * @author	Mario Hesse                       								*
%  * @version	v0.1.0                            								*
%  * @date		31.01.2019                        								*
%  ******************************************************************************

\section{Firmware}
	
Hier wird die Firmware oder Software beschrieben. Bilder von Programmabläufen, wie \autoref{pic:EinErstesBild}, können dabei nützlich sein.


\begin{figure}[htbp]
	\centering
	\includegraphics[width=0.5\linewidth]{pic/IntelligentesPlasma.pdf}
	\caption{Ein erstes Bild}
	\label{pic:EinErstesBild}
\end{figure}
	
	
	
	% Modultest/Messungen
	% -------------------
	\clearpage
	% !TeX spellcheck = de_DE

%  ******************************************************************************
%  * @file      tex/Modultest                                                   *
%  * @author    Mario Hesse                                                     *
%  * @version   v0.1.1                                                          *
%  * @date      16.10.2019                                                      *
%  ******************************************************************************

\section{Modultest/Messungen}

In diesem Abschnitt werden im Zuge der Arbeit erstellte Module getestet und Messungen dokumentiert. Mit dem Paket Tikz und PGF Plots kann man Daten direkt in Latex aufbereiten. Diagramme können in der \emph{tikzpicture} Umgebung erstellt werden. Typischerweise werden Diagramme in eine \emph{figure} Umgebung eingebettet, damit man sie wie gewöhnliche Abbildung behandeln kann.

\begin{figure}[htbp]
	\centering
	
	\begin{tikzpicture} 					% Begin einer neuen Grafik
		\begin{axis}[							% Begin des Achsen- bzw. Plotbereichs
		axis lines = left,						% Die Achsen des Plots sind rechts und unten			
		xlabel = $x$,							% Beschriftung der X-Achse
		ylabel = {$f(x)$},						% Beschriftung der Y-Achse
		]
		
		\addplot[								% Definition 1. Plot
		color=red,								% Linienfarbe ist Rot
		mark=square,							% Marker werden als Kästchen gezeichnet
		only marks,								% Es sollen nur Marken ohne Linie angezeigt werden
		]
		coordinates {							% Datenpunkte (x, y)
			(1, 0.83)
			(3, 0.04)
			(5, -0.20)
			(7, 0.10)
			(9, 0.06)
			(11,-0.08)
			(13,0.04)
			(15,0.03)
			(17,-0.07)
		};
		\addlegendentry{Messpunkte}					% Eintrag wird zur Legende hinzugefügt
		
		
		\addplot [								% Definition 2. Plot
		domain=1:17, 								% X-Bereich in dem geplottet werden soll
		samples=100, 								% Anzahl an Einzelpunkten, die für den Plot berechnet werden
		color=blue,									% Linienfarbe ist blau
		]
		{sin(deg(x))/x};							% Zu plottende Funktion
		\addlegendentry{Gefittete Kurve}			% Eintrag wird zu Legende hinzugefügt
		
		\end{axis}							% Ende des Achsen- bzw. Plotbereichs
	\end{tikzpicture}
	
	\caption{Ein in Latex gerendertes Diagramm}
	\label{pic:LatexDiagramm}
\end{figure}



	
	% Diskussion
	% ----------
	\clearpage
	% !TeX spellcheck = de_DE
% !TeX encoding = UTF-8

%  ******************************************************************************
%  * @file      tex/Diskussion                                                  *
%  * @author    Mario Hesse                                                     *
%  * @version   v0.1.0                                                          *
%  * @date      31.01.2019                                                      *
%  ******************************************************************************

\section{Diskussion}


	\subsection{Fazit}
	
	Hier steht eine kurze rückblickende Zusammenfassung der Arbeit und die wichtigsten Ergebnisse der Arbeit werden klar dargestellt und unterstrichen.
	

	\subsection{Fehler und Probleme}
	
		\paragraph{Problem 1}
		Ein Problem, auf das man bei der Weiterentwicklung achten sollte.
		
		\paragraph{Problem 2} 
		Ein weiteres Problem, auf das man bei der Weiterentwicklung achten sollte.
		
		\dots
		
	
	
	\subsection{Ausblick}

		\paragraph{Erweiterung/Vision 1}
		Eine mögliche Erweiterung oder Vision für das bisher entwickelte Projekt.
		
		\paragraph{Erweiterung/Vision 2}
		Eine weitere mögliche Erweiterung oder Vision für das bisher entwickelte Projekt.
		
		\dots










	
	

	% **************************************************************************
	% Abspann
	% **************************************************************************
	
	% Literatur
	% ---------
	\clearpage
	\phantomsection
	\addcontentsline{toc}{section}{Literaturverzeichnis}
	\sectionmark{Literaturverzeichnis}
	\renewcommand{\refname}{Literaturverzeichnis}								% Literaturverzeichnis statt Literatur als Überschrift verwenden
	\bibliography{bib/Literatur} 												% nutzt Angaben in der Datei "bib/Literatur.bib" um ein Literaturverzeichnis zu erstellen
	
	
	% Anhang
	% ------
	\clearpage
	% !TeX spellcheck = de_DE
% !TeX encoding = UTF-8

%  ******************************************************************************
%  * @file      tex/Anhang                                                      *
%  * @author    Mario Hesse                                                     *
%  * @version   v0.1.0                                                          *
%  * @date      31.01.2019                                                      *
%  ******************************************************************************



% ------
\appendix % Ändert die Nummerierung

% Schaltpläne
% -----------

\includepdf[pagecommand={\section{Schaltpläne} \label{sec:Schaltplaene} \subsection{Arduino} \label{sec:sch-Arduino}}, angle=90, scale=0.9]{app/Arduino-sch.pdf}




% Platinenlayout
% --------------

\includepdf[pagecommand={\section{Boardlayouts} \label{sec:Platinenlayout} \subsection{Arduino} \label{sec:brd-Arduino}}, angle=90, scale=0.8]{app/Arduino-brd.pdf}



% Verbindung der Platinen
% -----------------------
%\section{Verbindung der Platinen}



% Messdaten
% ---------
%\section{Messdaten}



% Datenblätter
% ------------
%\section{Datenblätter}



% Inhaltsverzeichnis der beiliegenden CD
% --------------------------------------
\clearpage 
%\includepdf[pagecommand={\section{Inhaltsverzeichnis der beiliegenden CD} \mario{Inhaltsverzeichnis CD fertig machen !}}, scale=0.7]{chapter/InhaltsverzeichnisCd.pdf}

\section{Inhaltsverzeichnis der beiliegenden CD}

\begin{forest}
	for tree={
		font=\ttfamily,
		grow'=0,
		child anchor=west,
		parent anchor=south,
		anchor=west,
		calign=first,
		edge path={
			\noexpand\path [draw, \forestoption{edge}]
			(!u.south west) +(7.5pt,0) |- node[fill,inner sep=1.25pt] {} (.child anchor)\forestoption{edge label};
		},
		before typesetting nodes={
			if n=1
			{insert before={[,phantom]}}
			{}
		},
		fit=band,
		before computing xy={l=15pt},
	}
	[	
		[1-Masterarbeit
			[Abbildungen]
			[Masterarbeit.pdf]
		]
		[2-Hardware
			[Datenblätter]
			[Eagle
				[Schaltung 1]
				[Schaltung 2]
				[\dots]
			]
		]
		[3-Firmware
			[CubeMX
				[Config File 1]
				[\dots]
			]
			[SystemWorkbench
				[Firmware 1]
				[\dots]
			]
		]
		[4-Extras
			[\dots]
		]
	]
\end{forest}

	
	
	% Eidesstattliche Versicherung
	% ----------------------------
	% !TeX spellcheck = de_DE

%  ******************************************************************************
%  * @file      tex/EidesstattlicheVersicherung                                 *
%  * @author    Mario Hesse                                                     *
%  * @version   v0.1.0                                                          *
%  * @date      31.01.2019                                                      *
%  ******************************************************************************

\clearpage
\thispagestyle{empty}
\phantomsection
\addcontentsline{toc}{section}{Eidesstattliche Versicherung}
\section*{Eidesstattliche Erklärung}

\smallskip
Ich erkläre hiermit, dass ich diese Abschlussarbeit selbstständig ohne Hilfe Dritter und ohne Benutzung anderer als der angegebenen Quellen und Hilfsmittel verfasst habe. Alle den benutzten Quellen wörtlich oder sinngemäß entnommenen Stellen sind als solche einzeln kenntlich gemacht.

\smallskip
\noindent
Diese Arbeit ist bislang keiner anderen Prüfungsbehörde vorgelegt und auch nicht veröffentlicht worden.

\bigskip
\noindent
Göttingen, den \today

\vspace{4em}
Vorname Name



	
	
	
% ***********************************************
\end{document}
% ***********************************************
