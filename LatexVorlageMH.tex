% Hintergrund: Bachelorarbeit im Studiengang Elektro- und Informationstechnik 
% an der Fakult�t Elektro- und Informationstechnik an der HTWK Leipzig 
% Stand: August 2014
% Aus einer Vorlage von Prof. Dr. Martin Gr�ttm�ller (HTWK Leipzig - Fakult�t IMN) 


\documentclass[a4paper,12pt]{scrartcl} % Seiteneinstellung und Gliederungseinstellung
\usepackage[latin1]{inputenc}
\usepackage{pdfpages}					% erm�glicht das Einbinden von PDFs

\usepackage{parskip}					% setzt einen Abstand nach einem \par Absatz
\usepackage{float}						% notwendig f�r Gleitubgebungen z.B. Figure
\usepackage{makeidx}					% anlegen eines automatischen Inhaltsverzeichnis
\usepackage{amsmath,amsfonts,amsthm}	% standard Krempel
\usepackage{tikz}						% Diagramme zeichnen
\usepackage[ngerman]{babel}				% Sprachpaket 
\usepackage[T1]{fontenc}				% Umlaute bei \hyphenation verwenden k�nnen (Silbentrennung)
\usepackage[square,numbers]{natbib}		% Literatur zitieren
\bibliographystyle{dinat}				% Literatur und Zitate
\usepackage{tocstyle}					% Punkte im Inhaltsverzeichnis
\usetocstyle{allwithdot} 

\setlength{\parindent}{0in} 			% Absatz, erste Zeile wird global NICHT einger�ckt


\usepackage{graphicx}					% erlaubt das Einbinden von Grafiken
\usepackage{subfig}						% erlaubt das Verwenden von Subfigures


\usepackage[headsepline]{scrpage2}		% Kopfzeile
\usepackage[
	    colorlinks=true,
	    urlcolor=blue,
	    citecolor=blue,
	    linkcolor=black
		]{hyperref}						% Verlinkung aller \ref,url,etc.


\usepackage{epstopdf}					% einbinden von eps Bilddateien

\usepackage{caption}					% erm�glicht die Beschriftung von Gleitobjekten
\usepackage{nameref}					% erlaubt Namensreferenzen
\usepackage[onehalfspacing]{setspace}	% Setzt das Dokument auf 1,5x Zeilenabstand
\usepackage{wasysym}					% zus�tzliche Symbole verwenden (Checkbox)
\usepackage{pgfplots, pgfplotstable}	% Diagramme zeichnen
\pgfplotsset{compat = newest}			% Einstellung f�r Diagramme
\usepackage{siunitx}                    % SI-Einheiten
\usepgfplotslibrary{units}				% Einheiten in Diagrammen plotten

\usepackage{longtable}					% lange Tabellen verarbeiten

\expandafter\def\expandafter\UrlBreaks\expandafter{\UrlBreaks\do\a		% Erlaubt einen Umbruch in einer URL 
	\do\b\do\c\do\d\do\e\do\f\do\g\do\h\do\i\do\j\do\k\do\l\do\m\do\n	% an allen angegebenen Stellen
	\do\o\do\p\do\q\do\r\do\s\do\t\do\u\do\v\do\w\do\x\do\y\do\z\do\A	
	\do\B\do\C\do\D\do\E\do\F\do\G\do\H\do\I\do\J\do\K\do\L\do\M\do\N	
	\do\O\do\P\do\Q\do\R\do\S\do\T\do\U\do\V\do\W\do\X\do\Y\do\Z\do\1
	\do\2\do\3\do\4\do\5\do\6\do\7\do\8\do\9\do\0\do\&} 	 

%\sloppy								% Nachl�ssige Silbentrennung, Blocksatzoptimiert
\hyphenation{OpenOCD}					% Worte, die NICHT oden nur auf bestimmte Weise getrennt werden sollen


% \paragraph mit Zeilenumbruch versehen ---------------------------
\makeatletter
\renewcommand\paragraph{\@startsection{paragraph}{4}{\z@}%
	{-3.25ex\@plus -1ex \@minus -.2ex}%
	{1.5ex \@plus .2ex}%
	{\normalfont\normalsize\bfseries}}
\makeatother
%------------------------------------------------------------------

\clubpenalty = 10000 % schliesst Schusterjungen aus
\widowpenalty = 10000 
\displaywidowpenalty = 10000% schliesst Hurenkinder aus 



% Beginn des Dokuments --------------------------------------------
\begin{document}

\ohead{\pagemark}
\ihead{\sectionmark}
\cfoot{}
\pagestyle{scrheadings}
\renewcommand{\sectionmark}[1]{\markright{#1}{}}

% Titelseite ------------------------------------------------------------------
% Titelseite ------------------------------------------------------------------
\begin{titlepage}
	
	\begin{center}
		\includegraphics[height=4cm]{Bilder/HTWK_Logo_CMYK}
		
		\vspace*{3cm}
		
		{\sc \LARGE Ein wirklich gutes Thema\\ auf zwei Zeilen}\\
		\vspace{1cm}
		{\sc \Large Belegarbeit}\\
	\end{center}
	
	\vspace{8cm}
	
	\begin{tabbing}
		\hspace{4cm}\=\kill
		Vorgelegt von:		\> Mario Hesse \\
		Eingereicht bei:	\> \dots \\ 
		Im Studienfach:		\> \dots \\
		Ort,Datum:			\> Leipzig, den \today
	\end{tabbing} 
	
\end{titlepage}

% Sperrvermerk ------------------------------------------------------------------------

%\thispagestyle{empty} % no page number
%\section*{Sperrvermerk}
%
%\medskip
%\noindent
%\emph{
%Die vorliegende Bachelorarbeit beinhaltet interne vertrauliche Informationen des Laboratory for Biosignal Processing (LaBP). Die Weitergabe des Inhaltes der Arbeit und eventuell beiliegender
%Zeichnungen und Daten im Gesamten oder in Teilen ist grunds�tzlich untersagt. Es d�rfen keinerlei Kopien oder Abschriften -- auch in digitaler Form -- gefertigt werden. Ausnahmen bed�rfen der vorherigen schriftlichen Genehmigung des Laboratory for Biosignal Processing (LaBP).\\[1cm]
%Laboratory for Biosignal Processing\\
%Dr.-Ing. Gerold Bausch\\
%Eilenburger Stra�e 13\\
%04317 Leipzig\\
%Telefon: +49 341 3076 3103 \\
%E-Mail: gerold.bausch@htwk-leipzig.de\\[2cm]
%Leipzig, den 3.Oktober 2015\\[2,5cm]
%\begin{tabbing}
%\hspace*{3cm}   \= \hspace*{6cm} \= \kill
% \> Gerold Bausch \> Mario Hesse \\	
%\end{tabbing}
%}




% Vorwort, Danksagung ------------------------------------------------------------------------
\clearpage
\phantomsection
\addcontentsline{toc}{section}{Vorwort}
\section*{Vorwort}

Pers�nliche Verbindung zum Thema. Grund f�r diesen Artikel. 
	

% Danksagung ------------------------------------------------------------------------------------
\clearpage
\phantomsection
\addcontentsline{toc}{section}{Danksagung}
\section*{Danksagung}

Bei gr��eren Arbeiten bedankt man sich bei Menschen und Institutionen, die Ma�geblich an der Entstehung teil hatten. (beim Bedanken Rangfolge beachten, wem geb�hrt der gr��te Dank) \\


% Kurzfassung, Abstract ------------------------------------------------------------------------
\clearpage
\phantomsection
\addcontentsline{toc}{section}{Kurzfassung, Abstract}
\section*{Kurzfassung}

Hier wird der Inhalt der Arbeit in ein paar Zeilen kurz dargestellt. Man soll auf Grund dieser Angabe entscheiden k�nnen, ob es lohnt die Arbeit zu lesen oder nicht.

\medskip
\noindent
Schlagw�rter: �berblick, Zusammenfassung, Einblick \\

 
\section*{Abstract}

This is the same than in "Kurzfassung" just in english language. This part is used to internationalize an article. 

\medskip
\noindent
Keywords: summary, content, overview \\


% Inhaltsverzeichnis ------------------------------------------------------------------------
\clearpage
\phantomsection
\addcontentsline{toc}{section}{Inhaltsverzeichnis}
\tableofcontents


% Abbildungsverzeichnis ------------------------------------------------------------------------
\clearpage
\phantomsection
\addcontentsline{toc}{section}{Abbildungsverzeichnis}
\listoffigures


% Tabellenverzeichnis ------------------------------------------------------------------------
\clearpage
\phantomsection
\addcontentsline{toc}{section}{Tabellenverzeichnis}
\listoftables



\vfill
% Abk�rzungsverzeichnis ------------------------------------------------------------------------
\phantomsection
\addcontentsline{toc}{section}{Abk�rzungsverzeichnis}
\section*{Abk�rzungsverzeichnis}


\begin{tabbing}
\hspace{2cm}\=\kill
HTWK 	\> Hochschule f�r Technik, Wirtschaft und Kultur\\
k.A. 	\> keine Angabe \\
\end{tabbing} 

\vfill

% Einheitenverzeichnis --------------------------------------------------------------------------
\clearpage
\phantomsection
\addcontentsline{toc}{section}{Einheitenverzeichnis}
\section*{Einheitenverzeichnis}

\begin{tabbing}
	\hspace{2cm}\=\kill	
	
	$V$			\> Volt \\
	$A$			\> Ampere \\
	$\Omega$	\> Ohm\\
	
\end{tabbing} 


% Formelverzeichnis ------------------------------------------------------------------------
\clearpage
\phantomsection
\addcontentsline{toc}{section}{Formelverzeichnis}
\section*{Formelverzeichnis}

\begin{tabbing}
	\hspace{2cm}\=\kill
	
	$U$					\> Spannung (allgemein)\\
	$R$					\> Widerstand (allgemein)\\
	$I$					\> Strom (allgemein)\\
	
\end{tabbing} 
	

% Glossar ---------------------------------------------------------------------------------
\clearpage
\phantomsection
\addcontentsline{toc}{section}{Glossar}
\section*{Glossar}

\begin{longtable}{p{4cm}p{11cm}}
	
	
	Glossar			& \dots ist eine Tabelle in der Fachw�rter f�r die forliegenede Arbeit definiert sind.\\
\end{longtable}

\addtocounter{table}{-1}	% Die Tabelle "Glossar wird nicht mit ins Tabellenverzeichnis aufgenommen

% Dokumententext ------------------------------------------------------------------------

% Literatur  -----------------------------------------------------------------
\clearpage
\phantomsection
\addcontentsline{toc}{section}{Literaturverzeichnis}
\sectionmark{Literaturverzeichnis}
\bibliography{Literatur} % nutzt Angaben in der Datei "Literatur.bib"


% Eidesstattliche Versicherung ------------------------------------------------------------------------
\clearpage
\thispagestyle{empty}
\phantomsection
\addcontentsline{toc}{section}{Eidesstattliche Versicherung}
\section*{Eidesstattliche Erkl�rung}

\smallskip
Ich erkl�re hiermit, dass ich diese Bachelorarbeit selbstst�ndig ohne Hilfe Dritter und ohne Benutzung anderer als der angegebenen Quellen und Hilfsmittel verfasst habe. Alle den benutzten Quellen w�rtlich oder sinngem�� entnommenen Stellen sind als solche einzeln kenntlich gemacht.

\smallskip
\noindent
Diese Arbeit ist bislang keiner anderen Pr�fungsbeh�rde vorgelegt und auch nicht ver�ffentlicht worden.

\smallskip
\noindent
Ich bin mir bewusst, dass eine falsche Erkl�rung rechtliche Folgen haben wird.

\bigskip
\noindent
Leipzig, den 

\pagestyle{empty}
\newpage % Leerseite f�r Notizen
~

% ***********************************************
\end{document}
% ***********************************************

